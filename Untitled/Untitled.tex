% Options for packages loaded elsewhere
\PassOptionsToPackage{unicode}{hyperref}
\PassOptionsToPackage{hyphens}{url}
%
\documentclass[
]{article}
\usepackage{amsmath,amssymb}
\usepackage{iftex}
\ifPDFTeX
  \usepackage[T1]{fontenc}
  \usepackage[utf8]{inputenc}
  \usepackage{textcomp} % provide euro and other symbols
\else % if luatex or xetex
  \usepackage{unicode-math} % this also loads fontspec
  \defaultfontfeatures{Scale=MatchLowercase}
  \defaultfontfeatures[\rmfamily]{Ligatures=TeX,Scale=1}
\fi
\usepackage{lmodern}
\ifPDFTeX\else
  % xetex/luatex font selection
\fi
% Use upquote if available, for straight quotes in verbatim environments
\IfFileExists{upquote.sty}{\usepackage{upquote}}{}
\IfFileExists{microtype.sty}{% use microtype if available
  \usepackage[]{microtype}
  \UseMicrotypeSet[protrusion]{basicmath} % disable protrusion for tt fonts
}{}
\makeatletter
\@ifundefined{KOMAClassName}{% if non-KOMA class
  \IfFileExists{parskip.sty}{%
    \usepackage{parskip}
  }{% else
    \setlength{\parindent}{0pt}
    \setlength{\parskip}{6pt plus 2pt minus 1pt}}
}{% if KOMA class
  \KOMAoptions{parskip=half}}
\makeatother
\usepackage{xcolor}
\usepackage[margin=1in]{geometry}
\usepackage{graphicx}
\makeatletter
\def\maxwidth{\ifdim\Gin@nat@width>\linewidth\linewidth\else\Gin@nat@width\fi}
\def\maxheight{\ifdim\Gin@nat@height>\textheight\textheight\else\Gin@nat@height\fi}
\makeatother
% Scale images if necessary, so that they will not overflow the page
% margins by default, and it is still possible to overwrite the defaults
% using explicit options in \includegraphics[width, height, ...]{}
\setkeys{Gin}{width=\maxwidth,height=\maxheight,keepaspectratio}
% Set default figure placement to htbp
\makeatletter
\def\fps@figure{htbp}
\makeatother
\setlength{\emergencystretch}{3em} % prevent overfull lines
\providecommand{\tightlist}{%
  \setlength{\itemsep}{0pt}\setlength{\parskip}{0pt}}
\setcounter{secnumdepth}{-\maxdimen} % remove section numbering
\ifLuaTeX
  \usepackage{selnolig}  % disable illegal ligatures
\fi
\usepackage{bookmark}
\IfFileExists{xurl.sty}{\usepackage{xurl}}{} % add URL line breaks if available
\urlstyle{same}
\hypersetup{
  pdftitle={RESIDÊNCIA E MOVIMENTAÇÃO: HÁ ALGUM PADRÃO PARA PEIXES EM RIACHO DE CABECEIRA?},
  hidelinks,
  pdfcreator={LaTeX via pandoc}}

\title{RESIDÊNCIA E MOVIMENTAÇÃO: HÁ ALGUM PADRÃO PARA PEIXES EM RIACHO
DE CABECEIRA?}
\author{true \and true \and true \and true}
\date{}

\begin{document}
\maketitle

\textbf{Resumo}

Estudos em riachos que investigam padrões de movimentação de peixes são
escassos e apresentam lacunas. As taxas de recaptura são baixas e os
potenciais fatores causadores são: baixa eficiência da captura,
mortalidade dos indivíduos marcados, alta mobilidade e perda de marcas.
O estudo objetivou investigar a comunidade de peixes em um arroio
estreito e de baixa vazão para aumentar a eficácia da captura, em escala
temporal e espacial menor para compreender os movimentos. O estudo
ocorreu nas corredeiras de um tributário do rio Paranhana, parte
superior da bacia do Rio dos Sinos no Município de Igrejinha,
totalizando 540m investigados. Os peixes foram coletados com pesca
elétrica, anestesiados com solução de Eugenol e marcados com VIFE. Em
cada evento, uma marca específica foi aplicada. Foram marcados 520
indivíduos com comprimento total maior que 5 cm, e recapturados com
marca 131 (25,2\%). Sendo 21,2\% recapturados no local da marcação,
1,3\% em corredeiras a jusante e 2,7\% a montante. As espécies com maior
quantidade de indivíduos recapturados foram C\emph{haracidium
pterostictum} e \emph{Heptapterus mustelinu}s. Os movimentos de \emph{C.
pterostictum} a montante ocorreram significativamente mais
frequentemente do que a jusante do ponto de marcação (chi2=5,4;
p=0,001). O padrão das recapturas indicou um comportamento parcialmente
residente na corredeira da marcação. Um indivíduo de \emph{H.
mustelinus} apresentou 4 marcas consecutivas. As marcações se
distribuíram uniformemente durante as campanhas, demonstrando a eficácia
do método. Os 75\% não recapturados podem ter se dispersado de forma
ativa, ou fora do local do estudo. O padrão é semelhante a outros
estudos, demonstrando que uma parte da comunidade é parcialmente
residente e outra altamente móvel.

Palavras chaves: - Marcação - Recaptura - \emph{C. pterostictum} -
\emph{H. mustelinus}

\begin{center}\rule{0.5\linewidth}{0.5pt}\end{center}

\textbf{INTRODUÇÃO}

A dinâmica de dispersão é caracterizada pelo movimento dos indivíduos
para longe do local de origem (LIDICKER e STENSETH, 1992) e é um fator
determinante para entender a dinâmica populacional, o comportamento, a
genética e a evolução das espécies (MCMAHON e MATTER, 2006), sendo
fundamental para compreensão das funções das populações (TILMAN et al.,
1997). A compreensão destes processos é essencial para a conservação dos
ecossistemas aquáticos (LUCAS e BARRAS, 2001). O movimento pode estar
relacionado com os estágios ontogenéticos como reprodução (HENRIQUES et
al., 2010) refúgio e evasão a predação (ROBERTS e ANGERMEIER, 2007). Os
distúrbios e pressões antropogênicas podem afetar ou impulsionar a
dispersão (SHUKLA E BHAT, 2018). Os movimentos podem ser de curta ou
longa distância. Os curtos são relativos a alterações nas condições
abióticas e disponibilidade de recursos (ALBANESE et al., 2004) enquanto
os de longa distância são relativos à reprodução e colonização
(HENRIQUES et al., 2010). A maioria dos estudos avaliam a migração de
peixes e são desenvolvidas em ambientes temperados e com peixes de
grande porte e com valor comercial (KENNEDY et al., 2013). Mesmo
apresentando um alto valor ecológico, peixes de pequeno porte e com
menor valor econômico são pouco estudadas (CAROLSFELD et al., 2003). A
migração mais extrema conhecida e documentada até então, é realizada
pela Brachyplatystoma rousseauxii, que desova no extremo oeste da
Amazônia e apresenta um ciclo migratório de aproximadamente 11.600 km
(BARTHEM et al., 2017). Em riachos, a movimentação dos peixes vem sendo
investigada e documentada, porém ainda apresenta lacunas. Thompson
(1933) calculou constantes de migração para espécies de peixes,
percebendo movimentos aleatórios. Posteriormente Gerking (1950) buscava
respostas sobre a estabilidade de populações em córregos no verão, e
anos depois, propôs o ``paradigma dos movimentos restritos'' (PMR)
(1959), determinando que a maioria das espécies de riachos são
sedentárias. Estudos mais recentes apontam que comunidades são compostas
tanto de indivíduos sedentários, quanto móveis (MAZZONI et al., 2018). A
maneira com que os peixes se movem nos riachos também influencia na
estrutura e na função das redes alimentares (WINEMILLER e JEPSEN, 1998).
Fausch et al.~(2002) explica que a dinâmica do movimento, resiliência,
ou troca de habitat pode ser difícil de se mensurar, uma vez que a
escala temporal no qual um organismo percebe seu ambiente nem sempre é
aparente para os pesquisadores. Riachos apresentam ambientes sequenciais
de corredeira (águas rasas, rápidas e ambiente rochoso) e poço (água
lenta com substrato fino) (TERESA E CASSATI, 2012). A complexidade e as
características geomorfológicas e hidrológicas do riacho influenciam
diretamente a composição funcional dos peixes (TERESA et al., 2015). Em
corredeiras, as espécies tipicamente encontradas são aquelas associadas
ao substrato, como Heptapterus sp., Rineloricaria sp., 3 Trichomycterus
sp. e Hemiancistrus sp. Em termos de abundância, a família Loricariidae
é que melhor caracteriza a ictiofauna de corredeiras (BECKER, 2002).
Estudos que investigam padrões de movimentação em riachos de cabeceira
são ainda menores, devido à dificuldade de realizar ensaios de marcação
e recaptura e ainda por apresentarem alta rotatividade de espécies
(BARBOSA, 2019). Além disso, as taxas de recaptura são baixas, como no
trabalho de Barbosa (2019), resultando uma média de 10\% de recaptura
dos indivíduos marcados. Em estudos abrangentes utilizando diversas
espécies, as taxas de recapturas variam entre 4,7\% (dados não
publicados) até 35\% (Mazzoni et al., 2018). Estes valores baixos podem
ser causados por quatro fatores: a) Mortalidade alta dos indivíduos
causado pelo manejo (handling) durante a captura e marcação; b) Perda de
marcas; c) Eficiência baixa do método de captura; d) Alta mobilidade dos
indivíduos, que leva a dispersão para fora da área de amostragem; As
possibilidades a e b foram testadas no laboratório (Leal et al., 2012,
Brennan et al., 2006). Os resultados destas publicações mostram, que com
o manejo apropriado durante a captura e marcação a taxa de mortalidade é
aproximadamente zero. Quanto a persistência das marcas, a durabilidade e
fragmentação depende do local aplicado e da espécie estudada. Nos
estudos, as marcas ficaram visíveis por mais de 70 dias, comprovando a
eficácia do método em curto prazo. Nesse contexto, o objetivo do
presente trabalho é executar ensaios de marcação e recaptura de peixes
em um riacho de cabeceira na zona de corredeira para compreender os
movimentos, em escala temporal e espacial menores para buscar
compreender a eficácia do método de amostragem.

\subsection{Referências
Bibliográficas}\label{referuxeancias-bibliogruxe1ficas}

Albanese, B., Angermeier, P.L. \& Dorai-Raj, S., 2004. Ecological
correlates of fish movement in a network of Virginia streams. Canadian
Journal of Fisheries and Aquatic Sciences 61: 857- 869

BARBOSA, Amanda. Efeito da estrutura de riachos de cabeceiras sobre a
organização da comunidade de peixes. 2019. 80 f.~Tese (Doutorado em
Biologia) - Programa de Pós-Graduação Biologia, Universidade do Vale do
Rio dos Sinos (UNISINOS), São Leopoldo, 2019.

BARTHEM, Ronaldo B. et al.~Goliath catfish spawning in the far western
Amazon confirmed by the distribution of mature adults, drifting larvae
and migrating juveniles. Scientific reports, v. 7, n.~1, p.~1-13, 2017.
13

BECKER, Fernando Gertum. Distribuição e abundância de peixes de
corredeiras e suas relações com características de hábitat local, bacia
de drenagem e posição espacial em riachos de Mata Atlântica (bacia do
rio Maquiné, RS, Brasil. 2002.

BOWER, Luke M. et al.~Effects of hydrology on fish diversity and
assemblage structure in a Texan coastal plains river. Transactions of
the American Fisheries Society, v. 148, n.~1, p.~207-218, 2019.

BRENNAN, Nathan P.; LEBER, Kenneth M.; BLACKBURN, Brett R. Use of
codedwire and visible implant elastomer tags for marine stock
enhancement with juvenile red snapper Lutjanus campechanus. Fisheries
Research, v. 83, n.~1, p.~90-97, 2007.

BUCKUP, P. A. et al.~Waterfall climging in Characidium (Crenuchidae:
Characidiinae) from eastern Brazil. Ichthyological Exploration of
Freshwaters, v. 11, n.~3, p.~273-278, 2000.

CAROLSFELD, Joachim et al.~Migratory fishes of South America. World
Fisheries Trust: Victoria, BC, Canada, 2003.

FAUSCH, Kurt D. et al.~Landscapes to riverscapes: bridging the gap
between research and conservation of stream fishes: a continuous view of
the river is needed to understand how processes interacting among scales
set the context for stream fishes and their habitat. BioScience, v. 52,
n.~6, p.~483-498, 2002.

Gerking, S. D. (1950). Stability of a stream fish population. The
Journal of Wildlife Management, 14(2), 193-202. Gerking, S. D. (1959).
The restricted movement of fish populations. Biological reviews, 34(2),
221-242.

GOWAN, Charles; FAUSCH, Kurt D. Why do foraging stream salmonids move
during summer?. Environmental Biology of Fishes, v. 64, n.~1-3,
p.~139-153, 2002. Gowan, C., Young, M. K., Fausch, K. D., \& Riley, S.
C. (1994). Restricted movement in resident stream salmonids: a paradigm
lost?. Canadian Journal of Fisheries and Aquatic Sciences, 51(11),
2626-2637.

Henriques, R., Sousa, V., \& Coelho, M. M., 2010. Migration patterns
counteract seasonal isolation of Squalius torgalensis, a critically
endangered freshwater fish inhabiting a typical Circum-Mediterranean
small drainage. Conservation Genetics 11:1859-1870. 14

JONSSON, Bror; JONSSON, Nina. Partial migration: niche shift versus
sexual maturation in fishes. Reviews in Fish Biology and Fisheries, v.
3, n.~4, p.~348-365, 1993.

Krebs, Charles J. Ecological Methodology. 1999. KENNEDY, R. J. et
al.~Upstream migratory behaviour of wild and ranched Atlantic salmon
Salmo salar at a natural obstacle in a coastal spate river. Journal of
fish biology, v. 83, n.~3, p.~515-530, 2013 KNAEPKENS, G.; BAEKELANDT,
K.; EENS, M. Assessment of the movement behaviour of the bullhead
(Cottus gobio), an endangered European freshwater fish. Animal Biology,
v. 55, n.~3, p.~219-226, 2005. Langerhans, R. B., \& Reznick, D. N.,
2010. Ecology and evolution of swimming performance in fishes:
predicting evolution with biomechanics. Fish locomotion: an
eco-ethological perspective 220:248. LARENTIS, Crislei et al.~Fauna de
peixes em riachos: avaliação das intervenções antrópicas sobre os
atributos e estrutura funcional das assembleias. 2015. LEAL, Mateus
Evangelista; BARBOSA, Amanda Saldanha; SCHULZ, Uwe Horst. Uso de
Implante Visual Fluorescente de Elastômero (VIFE) na marcação de
pequenos peixes de água doce tropicais. Biotemas, v. 25, n.~3,
p.~311-315, 2012. LIDICKER, W. Z.; STENSETH, N. C. To disperse or not to
disperse: who does it and why?. In: Animal dispersal. Springer,
Dordrecht, 1992. p.~21-36. Lucas, M. C., \& Baras, E. (2001). Migration
of freshwater fishes.Blackwell Science. LUCENA, C. A. S. et al.~O uso de
óleo de cravo na eutanásia de peixes. Boletim Sociedade Brasileira de
Ictiologia, v. 105, p.~20-24, 2013. Mazzoni, R., Pinto, M. P.,
Iglesias-Rios, R., \& Costa, R., 2018. Fish movement in an Atlantic
Forest stream. Neotropical Ichthyology 16. MAZZONI, R.; MENDONÇA, R. S.;
CARAMASCHI, E. P. Reproductive biology of Astyanax janeiroensis
(Osteichthyes, Characidae) from the Ubatiba River, Maricá, RJ, Brazil.
Brazilian Journal of Biology, v. 65, n.~4, p.~643-649, 2005. McMahon, T.
E., \& Matter, W. J., 2006. Linking habitat selection, emigration and
population dynamics of freshwater fishes: a synthesis of ideas and
approaches. Ecology of Freshwater Fish 15:200-210. Radinger, J., \&
Wolter, C., 2014. Patterns and predictors of fish dispersal in rivers.
Fish and fisheries 15:456-473. 15 Roberts, J. H., \& Angermeier, P. L.,
2007. Movement responses of stream fishes to introduced corridors of
complex cover. Transactions of the American Fisheries Society
136:971-978. RODRÍGUEZ, Marco A. Restricted movement in stream fish: the
paradigm is incomplete, not lost. Ecology, v. 83, n.~1, p.~1-13, 2002.
Schlosser IJ, Angermeier PL (1995) Spatial variation in demographic
processes of lotic fishes: conceptual models, empirical evidence and
implications for conservation. Am Fish Symp No 17:392--401 Schulz, U.
H., Nabinger, V., \& Gomes, L. P. (2006). Relatório Final do Projeto
Monalisa. São Leopoldo, RS. Comitê de gerenciamento da bacia do Rio dos
SinosCOMITESINOS, 18p Shukla, R., \& Bhat, A. (2018). Nestedness
patterns and dispersal dynamics in tropical central Indian stream fish
metacommunities. Freshwater Science, 37(1), 147-158. SILVA, Raquel Costa
da. Movimento longitudinal de peixes: uma análise cienciométrica,
manutenção de comunidades e implicação na colonização de um riacho
costeiro. 2013. TERESA, Fabrício B.; CASATTI, Lilian. Influence of
forest cover and mesohabitat types on functional and taxonomic diversity
of fish communities in Neotropical lowland streams. Ecology of
Freshwater Fish, v. 21, n.~3, p.~433-442, 2012. THIEM, Jason D. et
al.~Hypoxic conditions interrupt flood‐response movements of three
lowland river fish species: Implications for flow restoration in
modified landscapes. Ecohydrology, v. 13, n.~3, p.~e2197, 2020. TERESA,
Fabrício Barreto; CASATTI, Lilian; CIANCIARUSO, Marcus Vinicius.
Functional differentiation between fish assemblages from forested and
deforested streams. Neotropical Ichthyology, n.~ahead, p.~00-00, 2015.
Thompson, D. H. (1933). The migration of Illinois fishes. Biological
notes; no. 001. Tilman, D., Lehman, C. L., \& Kareiva, P. (1997).
Population dynamics in spatial habitats. Spatial ecology: The role of
space in population dynamics and interspecific interactions, 3-20.
Winemiller, K. O., \& Jepsen, D. B. (1998). Effects of seasonality and
fish movement on tropical river food webs. Journal of fish Biology, 53,
267-296

\end{document}
